\documentclass[12pt,oneside]{article}
\usepackage[T1]{fontenc}
\usepackage{latexsym}
\usepackage[activeacute,spanish]{babel}
\usepackage{amsfonts}
\usepackage{amsmath}
\usepackage{amssymb}
\usepackage{amsthm}
\usepackage{graphicx}
\usepackage[all]{xy}
\usepackage{tikz}
\usepackage[normalem]{ulem}
\usepackage{cancel}
\usepackage{soul}
\usepackage[retainorgcmds]{IEEEtrantools}
\usepackage{mathrsfs}
\usepackage{makeidx}
\newtheorem{prob}{Problema}
\addtolength{\hoffset}{-2cm}
\addtolength{\textwidth}{4cm}
\addtolength{\voffset}{-2.5cm}
\addtolength{\textheight}{5cm}
\pagestyle{empty}


\begin{document}


\begin{flushright}
{\large\textbf{Tarea 2} \textnormal{Cova Pacheco Felipe de Jes\'us}, \textnormal{Domingo 6 de septiembre de 2015}}
\end{flushright}

\begin{prob} %[Fuente, capítulo, ejercicio.]
Encuentra los enteros positivos menores o iguales a $3076$ que son divisibles por $23$.
\end{prob}

\begin{proof}
Aplicando el algoritmo de la divisi\'on para $3076$ y $23$ obtenemos que $3076 = 23 * 133 + 17$\\
Por lo tanto $23 * i$ es un entero positivo menor a $3076$ divisible por 23, para cada $i \in 20,...,133\}$\\\\
\end{proof}


\begin{prob} %[Fuente, capítulo, ejercicio.]
Encuentra los enteros positivos que est\'an entre $1976$ y $3776$ y que son divisibles entre $13$ y $15$.
\end{prob}

\begin{proof}
Para que un n\'umero sea divisible entre $13$ y $15$, debe ser divisible por su producto $13*15=195$.\\
aplicando el algoritmo de la divisi\'on para $3776$ y $195$ obtenemos que $3776 = 195*19+71$\\
El m\'inimo m\'ultiplo de $195$ mayor que $1976$ es: $195 * 11$\\
Por lo tanto Los n\'umeros entre $1976$ y $3776$ divisibles por $13$ y $15$ son $195*i$ con $i \in \{11,...,19\}$ \\\\
\end{proof}


\begin{prob} %[Fuente, capítulo, ejercicio.]
Considera que $a, b$ y $c$ son enteros. Demuestra o da un contraejemplo:\\
a. Si $a^2 =b^2$ entonces $a=b$.\\
b. Si $a|b$ y $b|a$ entonces $a=b$.\\
c. Si $a|bc$ entonces $a|b$ y $a|c$.
\end{prob}

\begin{proof}
a) Falso, sea $a = -3$ y $b = 3$\\
entonces $(-3)^2 = (9) = (3)^2$, pero $-3 \neq 3$.\\\
b) Falso, $3|-3$ y $-3|3$ pero $-3 \neq 3$\\\
c) Falso, $6|6*3$ pero $6\nmid3$.\\\\
\end{proof}



\begin{prob} %[Fuente, capítulo, ejercicio.]
Prueba que cualesquiera dos enteros consecutivos son primos relativos.
\end{prob}

\begin{proof}
Consideramos $(x, x+1) = d$ y suponemos que $d \geq 2$.\\
entonces $d|x$ y $d|x+1$, es decir, existen\\
$m,n \in \mathbb{N}$ tales que $x=dm$ y $x+1 = dn$.\\
Como $x+1 > x$, entonces $n \geq m+1$, lo que\\
implica que $nd \geq md + d$, as\'i $x+1 \geq x+d!$, $d \geq 2$\\
Entonces debe ocurrir que $d=1$.\\\\
\end{proof}

\begin{prob} %[Fuente, capítulo, ejercicio.]
Usando un razonamiento recursivo encuentra $[24, 28, 36, 40, 48]$.
\end{prob}

\begin{proof}
$[24,28,36,40,48] = [24,[28,36,40,48]] = [24,[28,[36,40,48]]]$\\
$= [24,[28,[36,[40,48]]]] = [24,[28,[36,240]]]$
$= [24,[28,720]] = [24,5040] = 5040$\\\\ 
\end{proof}

\begin{prob} %[Fuente, capítulo, ejercicio.]
Encuentra $[a,b]$, en los casos en que $a|b$, $b|a$, $a=1$ y $a=b$.
\end{prob}

\begin{proof}
$\bullet$ Si $a|b$, entonces $[a,b] = b$.\\
$\bullet$ Si $b|a$, entonces $[a,b] = a$.\\
$\bullet$ Si $a=1$, entonces $[a,b] = b$.\\
$\bullet$ Si $a=b$, entonces $[a,b] = a = b$.\\\\
\end{proof}


\begin{prob} %[Fuente, capítulo, ejercicio.]
Sean $a,b$ y $c$ n\'umero enteros. Demuestra las siguientes propiedades:\\
b. $(a,[b,c]) = [(a,b),(a,c)]$\\ 
c. $[a,(b,c)] = ([a,b],[a,c])$
\end{prob}

\begin{proof}
b) Sean\\
$a = P_1^{a_1} \cdots P_t^{a_t}$\\
$b = P_1^{b_1} \cdots P_t^{b_t}$\\
$c = P_1^{c_1} \cdots P_t^{c_t}$\\
entonces $(a,[b,c]) =$  
\begin{equation*}
\prod_{i=1}^t  P_i^{min\{a_i,max\{b_i,c_i\}\}} \end{equation*}\\
y $[(a,b), (a,c)]$ =  \begin{equation*} \prod_{i=1}^t  P_i^{max\{min\{a_i,b_i\}, min\{a_i,c_i\}\}} \end{equation*}\\
y como $min\{a_i, max\{b_i,c_i\}\} = max\{min\{a_i,b_i\}, min\{a_i,c_i\}\}$\\
para cualquier eleccio\'on de $a_i, b_i, c_i$, conclu\'imos\\
que $(a,[b,c]) = [(a,b), (a,c)]$.\\\

c) De manera an\'aloga al inciso anterior, utilizando\\
que $max\{a,min\{b,c\}\} = min\{max\{a,b\},max\{a,c\}\}$\\\\
\end{proof}


\begin{prob} %[Fuente, capítulo, ejercicio.]
Utilizando el algoritmo de Euclides expresa como combinaci\'on lineal el MCD de las siguientes parejas de enteros.\\
a. $1776$ y $3076$\\ 
b. $4076$ y $2076$
\end{prob}

\begin{proof}
$3076 = 1*1776 + 1300\\
1776 = 1*1300 + 476\\
1300 = 2*476 + 398\\
476 = 1*348 + 128\\
348 = 2*128 + 92\\
128 = 1*92 + 36\\
92 = 2*36 + 20\\
36 = 1*20 + 16\\
20 = 1*16 + 4\\
$Por lo tanto $(3076, 1776) = (16,4) = 4$.\\\\
\end{proof}


\begin{prob} %[Fuente, capítulo, ejercicio.]
Encuentra el n\'umero de ceros que se encuentran al final de $378!$.
\end{prob}

\begin{proof}
Sea $\alpha_2$ y $\alpha_5$ las potencias m\'as altas de $2$ y $5$ que dividen a $378!$ respectivamente. Claramente $\alpha_2 \geq \alpha_5$.\\
Encontremos $\alpha_5$\\
$
378 = 5*75 + 3\\
378 = 5^2*15 + 3\\
378 = 5^3*3 + 3\\
$Por lo tanto La potencia m\'as alta de $5$ que divide a $378$ es $93$.\\
Como $\alpha_2 \geq \alpha_5$ tenemos que $\alpha_{10} = 93$, que es el n\'umero de ceros al final de $378!$\\\\
\end{proof}


\begin{prob} %[Fuente, capítulo, ejercicio.]
Encuentra la potencia mas alta de $7$ que divide a $1001!$.
\end{prob}


\begin{proof}
Sea $\alpha_7$ la potencia m\'as alta de $7$ que divide a $1001!$\\
Notamos que:\\
$1001 = 7*143 + 0$\\
$1001 = 7^2*20 + 21$\\
$1001 = 7^3*2 + 315$\\
Por lo tanto $\alpha_7 = 143 + 20 + 2 = 165$.\\\\
\end{proof}


\begin{prob} %[Fuente, capítulo, ejercicio.]
Dec\'imos que un n\'umero $n$ es poderoso si siempre que $p$ es un factor primo de $n$, $p^2$ tambi\'en lo es.\\
a. Encuentra los primeros $3$ n\'umeros poderosos.\\
b. Muestra que cualquier n\'umero poderoso puede ser expresado de la forma $a^2b^3$, donde $a$ y $b$ son enteros positivos.
\end{prob}

\begin{proof}
Sea $n$ un n\'umero poderoso. Considerando la descomposici\'on en primos de $n$, podemos escribir.\\
$n = P_1^{n_1} \cdots P_r^{n_r} * P_{r+1}^{n_{r+1}} \cdots P_t^{n_t}$\\
de tal forma que $2|n_i$ para $i \in \{1,...,r\}$ y $2 \nmid n_i$ entonces $n_i = 2m_i + 1$ y como $n$ es poderoso sabemos que $n_i \geq 2$ para toda $i$.\\
entonces $m_i \geq 1$ y as\'i podemos poner a $n_i$ de la forma $n_i = 2(m_i -1)+3$.\\
definimos $a = P_1^{n_1/2} \cdots P_r^{n_r/2} * P_{r+1}^{(n_{r+1}-3)/2} \cdots P_t^{(n_i - 3)/2}$ y $b = P_{r+1} \cdots P_t$\\
De esta forma $a^2b^3 = (P_1^{n_1/2} \cdots P_r^{n_r/2} * P_{r+1}^{(n_{r+1}-3)/2} \cdots
P_{r+1}^{(n_{t}-3)/2})^2 (P_{r+1} \cdots P_t)^3$\\
$= P_1^{n_1} \cdots P_r^{n_r} * P_{r+1}^{n_{r+1}-3} \cdots P_t - 3 * P_{r+1}^3 \cdots P_t^3$\\
$= P_1^{n_1} \cdots P_t^{n_t} = n$.
\end{proof}

\end{document}




