\documentclass[12pt,oneside]{article}
\usepackage[T1]{fontenc}
\usepackage{latexsym}
\usepackage[activeacute,spanish]{babel}
\usepackage{amsfonts}
\usepackage{amsmath}
\usepackage{amssymb}
\usepackage{amsthm}
\usepackage{graphicx}
\usepackage[all]{xy}
\usepackage{tikz}
\usepackage[normalem]{ulem}
\usepackage{cancel}
\usepackage{soul}
\usepackage[retainorgcmds]{IEEEtrantools}
\usepackage{mathrsfs}
\usepackage{makeidx}
\newtheorem{prob}{Problema}
\addtolength{\hoffset}{-2cm}
\addtolength{\textwidth}{4cm}
\addtolength{\voffset}{-2.5cm}
\addtolength{\textheight}{5cm}
\pagestyle{empty}


\begin{document}


\begin{flushright}
{\large\textbf{Tarea 1} \textnormal{Felipe de Jes\'us Cova Pacheco}, \textnormal{Domingo 6 de septiembre de 2015}}
\end{flushright}

\begin{prob} %[Fuente, capítulo, ejercicio.]
Sean $(A, \leq)$ un conjunto parcialmente ordenado y $\emptyset \neq B \subseteq A$. Recordemos que una cota inferior de $B$ es un elemento $d \in A$ tal que $d \leq$ b para todo $b \in B$. El �nfimo de B es la mayor de las cotas inferiores de $B$, es decir, es una cota inferior $d_0$ de $B$ tal que $d \leq d_0$ para toda cota inferior $d$ de $B$. De manera complementaria se define el supremo de B como la menor de las cotas superiores de $B$. Un conjunto parcialmente ordenado $(A, \leq)$ es una
ret�cula si para cualesquiera elementos $a, b \in A$ el conjunto $\{a, b\}$ tiene tanto \'infimo como
supremo.\\\

Demuestre que:\\
i. Si $S \neq \emptyset$, entonces el conjunto potencia $\wp(S)$ junto con la inclusi\'on de conjuntos es una ret\'icula que, adem\'as tiene un \'unico elemento m\'aximo bajo la contenci\'on.\\
ii. De un ejemplo de un conjunto parcialmente ordenado que no sea una ret\'icula.\\
iii. De un ejemplo de una ret\'icula que no tenga elemento m\'aximo bajo la relaci\'on de orden
y un ejemplo de un conjunto parcialmente ordenado con dos elementos m\'aximos.
\end{prob}

\begin{proof}
i) Claramente $\wp(s) \neq 0 $ y la relaci\'on $\subseteq$ es reflexiva, transitiva y sim\'etrica\\
Por lo tanto $(\wp(s), \subseteq)$ es un COPO\\
Adem\'as para todo $x$ en $\wp(s) $ se tiene que $x \subseteq S$\\
Por lo tanto S es el \'unico elemento m\'aximo\\
Si $A, B$ en $\wp(s)$, entonces $\{A,B\}$ tiene a $A\cup B$ como supremo y a $A\cap B$ como \'infimo\\
Por lo tanto $\wp(s)$ es una ret\'icula.\\\

ii) Si $A = \{1,2,3\}$ y $\sim$ est\'a definida para $a,b \in A$\\
como $a \sim b$ entonces $a \leq b$\\
entonces $(a, \sim)$ es COPO\\\
pero $\{2,3\}$ no tiene supremo, por lo que no es ret�cula.\\\

iii) $- (\mathbb{N}, \leq )$ es una ret\'icula sin elemento m\'aximo\\
$- (\{1,2,3\}, a \mid b)$ es un COPO con dos elementos m\'aximos.\\\\
\end{proof}


\begin{prob} %[Fuente, capítulo, ejercicio.]
Una ret�cula $(A, \leq)$ se dice que es completa si cualquier subconjunto no vac\'io de $A$ tiene
tanto \'infimo como supremo. Una funci\'on de conjuntos parcialmente ordenados $f : A \rightarrow B$
se dice que preserva el orden si siempre que $a \leq_A  a'$ entonces $f(a) \leq_B f(b)$. Demuestre que
una funci\'on $f$ que preserva orden de una ret\'icula completa $A$ en s\'i misma tiene al menos un
elemento que se queda fijo (es decir, existe $a \in A$ tal que $f(a) = a$).
\end{prob}

\begin{proof}
Como $A \subseteq A$. y $A \neq \emptyset$\\
por ser $(A, \leq)$ completa, tenemos que A tiene \'infimo a\\
Sea $S = \{ x$ en $A \mid x \leq f(x)\}$\\
claramente $a \subseteq f(a)$ (pues $f(a) \in A$)\\
entonces $a$ en $S$ y $S \neq \emptyset$\\
Por ser $(A, \leq)$ completa y $S \neq \emptyset$\\
Sabemos que S tiene supremo s\\
Entonces para todo $x$ en $S, x \leq S$ como f presencia orden, se cumple que $x \leq f(x) \leq f(s)$\\
Entonces $f(s)$ es cota superior de $S$ y entonces $s \leq f(s)$\\
Adem\'as $f(s) \leq f(f(s))$ entonces $f(s)$ en $S$\\
y como s es supremo de S\\
se cumple que $f(s) \leq S$\\
Por lo tanto $f(s) = S$\\\\
\end{proof}



\begin{prob} %[Fuente, capítulo, ejercicio.]
Sea $S \subseteq  \mathbb{R}^2$ con $S = \{(x, y) \in  \mathbb{R}^2 \mid y \leq 0\}$. Defina el orden $\leq$ en $S$ como:\\
$(x_1, y_1) \leq (x_2, y_2)$ si y s\'olo si $x_1 = x_2$ y $y_1 \leq y_2$.\\
Demuestre que $(S, \leq)$ es un COPO y que $S$ tiene infinidad de elementos m\'aximos.
\end{prob}

\begin{proof}
Reflexividad:\\
Si $(X_1, Y_1)$ en $S$ entonces $X_1 = X_1$ y $Y_1 \leq Y_1$\\
Por lo tanto $(X_1, Y_1) \leq (X_1, Y_1)$\\
Transitividad:\\
Si  $(X_1, Y_1) \leq (X_2, Y_2)$ y $(X_2, Y_2) \leq (X_3, Y_3)$\\
entonces $X_1 = X_2 = X_3$ y $X_1 \leq X_2 \leq X_3$\\
Antisimetr\'ia:\\
si $(X_1, Y_1) \leq (X_2, Y_2)$ y $(X_2, Y_2) \leq (X_1, Y_1)$
entonces $X_1 = X_2$, $Y_1 \leq Y_2$, $X_1 = X_2$ y $Y_2 \leq Y_1$\\
Por lo tanto $(X_1, Y_1) = (X_2, Y_2)$\\
Entonces $(S, \leq)$ es COPO y para cada $x$ en $\mathbb{R}, (X,0)$ es un elemento m\'aximo de S\\\\
\end{proof}



\begin{prob} %[Fuente, capítulo, ejercicio.]
Un anillo $R$ donde todo elemento $a \in R$ cumple que $a^2 = a$ es llamado anillo Booleano.\\
Demuestre que:\\
a. Si R es anillo Booleano entonces R es conmutativo.\\
b. Si R es anillo Booleano entonces $a + a = 0$.\\
c. De un ejemplo de un anillo Booleano.
\end{prob}

\begin{proof}
a) Sean $x,y$ en $\mathbb{R}$\\
entonces $(x,y)^2 = x^2 + xy + yx + y^2$\\
y como R es booleano $(x,y)^2 = x + y$\\
entonces $x + xy + yx + y = x + y$\\
entonces $xy = -yx = yx$,            $(-yx)^2 = -yx = yx$\\
Por lo tanto R es conmutativo\\\

b) $ (a+a)^2 = a+a$ por ser R booleano\\
y $(a+a)^2 = a^2 + 2a^2 + a^2 = a+a+a+a$\\
de lo anterior se sigue que $a+a=0$\\\

c)$ - \mathbb{Z}_2, \mathbb{Z}_2 x \geq _2$\\
$- \forall S$ conjunto, $(\wp(s), \subseteq)$\\
$\mathbb{Z}_2 x \mathbb{Z}_2 \cong \wp(\{a,b\})$\\\\
\end{proof}

\begin{prob} %[Fuente, capítulo, ejercicio.]
Demuestre que un anillo finito con m\'as de un elemento y sin divisores de cero es un anillo
con divisi\'on.
\end{prob}

\begin{proof}
Sea $A$ con m\'as de un elemento y sea $a$ en $A$, $a \neq 0$\\
definimos $f_G: A$ entonces $A$, es inyectiva pues si $xa = ya$,\\
como $A$ no tiene divisores de cero y $a \neq 0$ conclu\'imos que $x = y$.\\
Dado que A es fin\'ito y $f_G$ inyectiva, tenemos que $f_a$ es biyectiva\\
entonces existe $x$ en $A$ tal que $a = az$.\\
ahora si $y$ en $A$, $y \neq 0$. Entonces existe $x \in A$ tal que\\
$xa = y$ lo que implica que $y = xa = xaz = yz$\\
Por lo tanto $A$ tiene elemento identidad por la derecha.\\
Analogamente $A$ tiene identidad izquierda y por lo tanto identidad e.\\
Finalmente, para cada $a$ en $A$, tenemos que $f_a$ es biyectiva,\\
entonces existe $x \in A$ tal que $f(x) = xa = e$, es decir $x = a^{-1}$
\end{proof}

\begin{prob} %[Fuente, capítulo, ejercicio.]
Sea R un anillo con m\'as de un elemento tal que para todo elemento no cero $a \in R$ existe un
\'unico $b \in B$ tal que $aba = a$. Demuestre que:\\
a. $R$ no tiene divisores de cero.\\
b. $bab = b$.\\
c. $R$ tiene identidad.\\
d. $R$ es un anillo con divisi\'on.
\end{prob}

\begin{proof}
\end{proof}


\begin{prob} %[Fuente, capítulo, ejercicio.]
Un elemento de un anillo es nilpotente si $a^n = 0$ para alguna $n \in  \mathbb{N}$. Demuestre que en un anillo conmutativo con $a, b \in R$ nilpotentes, $a + b$ es nilpotente.
\end{prob}

\begin{proof}
Sean $a,b \in A$ nilpotentes\\
entonces $a^m = 0$ y $b^n = 0$ p.a. $m,n \in \mathbb{N}$\\
entonces $(a+b)^{n+m} = \sum\limits_{i=1}^{m+n}  \-\binom{n+m}{i} a^i b^{m+n-i}$\\
si $i \leq m$ entonces $m+n-i \geq  n$ entonces\\
$a^i b^{m+n-i} = 0$\\
si $i \geq m$ entonces $a^i b^{m+n-i} = 0$\\
Por lo tanto $(a+b)^{m+n} = 0$\\
elemento m\'aximo de S\\\\
\end{proof}


\begin{prob} %[Fuente, capítulo, ejercicio.]
Demuestre que son equivalentes para R un anillo:\\
a. R no tiene elementos nilpotentes no cero.\\
b. Si $a \in R$ y $a^2 = 0$ entonces $a = 0$.
\end{prob}

\begin{proof}
a) implica b)\\
Sea $a \in R$ tal que $a^2 = 0$\\
 entonces $a$ es nilpotente por a)\\
tenemos que $a = 0$\\
b) implica a)\\
Sea a un elemento nilpotente de R\\
y sea n el m\'inimo entero tal que $a^n = 0$\\
Sup. que a $\neq 0$\\
- Si $n = 2k$, entonces $k \geq 1$ y $a^n = a^{2k} = (a^k)^2 = 0$\\
por b) tenemos que $a^k = 0$ !\\
(pues $k < n$)\\
- Si $n = 2k +1$ entonces\\
$a^n+1 = a^{2k + 2} = (a^{k+1})^2 = 0$\\
por b) tenemos que $a^{k+1} = 0$ !\\
(pues $k+1 < n$)\\
Por lo tanto $a = 0$\\\\
\end{proof}


\begin{prob} %[Fuente, capítulo, ejercicio.]
Demuestre que el conjunto:\\
$\{k - \frac{1}{n} \mid k,n \in  \mathbb{N}\}$\\
Satisface el Principio de Buen Orden.
\end{prob}

\begin{proof}
Sea $S \subseteq \{k - \frac{1}{n} \mid n,k \in \mathbb{N} \}$\\
entonces existen $N_1, N_2 \subseteq \mathbb{N}$\\
tales que $S \subseteq \{k - \frac{1}{n} \mid k \in N_1, n \in \mathbb{N}_2 \}$\\
como $\mathbb{N}$ satistace el PBO\\
sabemos que $N_1$ y $N_2$ tienen\\
primer elemento $n_1$ y $n_2$ respectivamente\\
Conclu\'imos que $n_1 - \frac{1}{n_2}$ es un elemento m\'inimo de S\\\\  
\end{proof}


\begin{prob} %[Fuente, capítulo, ejercicio.]
Demuestre que para cualquier $n \in  \mathbb{N}$ se cumple:
\begin{equation*}
\sum_{i=1}^n i(i!) = (n+1)! - 1
\end{equation*}
\end{prob}

\begin{proof}
Caso base $n=1$\\
\begin{equation*}
\sum_{i=1}^n i(i!) = 1*1! = 1 = (1+1)! -1 = 2 - 1=1\\
\end{equation*}
supongamos que se cumple para $n=k$\\
Sea n=k+1\\
entonces
\begin{equation*}
\sum_{i=1}^{k+1} i(i!) = \sum_{i=1}^k i(i)! + (k+1)(k+1)!\\
\end{equation*}
$=(k+1)! - 1 + (k+1)(k+1)! = (k+1)!(1+n+1)-1$\\
$=(n+2)! - 1$
\end{proof}


\begin{prob} %[Fuente, capítulo, ejercicio.]
Una pelota es soltada a una altura de $4$ metros y cada vez que toca el piso rebota a $\frac{3}{4}$
de la altura previa. �Cu\'al es la distancia total que la bola recorre (arriba y abajo) cuando llega a
la altura m\'axima de su d\'ecimo rebote?
\end{prob}

\begin{proof}
$4 + 2 \frac{4}{3} (4) + 2(\frac{4}{3})^2(4) + \dots + 2(\frac{4}{3})^9 (4) + (\frac{4}{3})^{10} 4 $\\\\
\end{proof}


\begin{prob} %[Fuente, capítulo, ejercicio.]
Al principio de cada a�o, se depositan $100$ pesos en una cuenta de ahorros. Al t\'ermino de
cada a�o, se paga el 5 \% de inter\'es del monto total existente en la cuenta al principio del a�o.
Obtenga una f\'ormula para calcular el monto total del dinero acumulado al inicio del a�o $n$.
\end{prob}

\begin{proof}
$X_0 = 100$\\
$X_1 = X_0 - X_0 (0.05) + 100$ entonces $X_n = X_{n-1}(1-0.05) + 100$\\
$X_2 = X_1 - X_1 (0.05) + 100$\\\

$X_n = (X_{n-1} (1 - 0.05) + 100) (1-0.05) + 100$\\
$X_n = (X_{n-2} (0.95) + 100) (0.95 + 100)$\\
$\vdots$\\
$X_n = 100(0.95)^4 + 100 (0.95)^{n-1} + \dots + 100(0.95) + 100$\\
=
\begin{equation*}
\sum_{i=0}^n (100)(0.95)^i
\end{equation*}
\end{proof}

\end{document}




