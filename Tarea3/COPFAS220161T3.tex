\documentclass[12pt,oneside]{article}
\usepackage[T1]{fontenc}
\usepackage{latexsym}
\usepackage[activeacute,spanish]{babel}
\usepackage{amsfonts}
\usepackage{amsmath}
\usepackage{amssymb}
\usepackage{amsthm}
\usepackage{graphicx}
\usepackage[all]{xy}
\usepackage{tikz}
\usepackage[normalem]{ulem}
\usepackage{cancel}
\usepackage{soul}
\usepackage[retainorgcmds]{IEEEtrantools}
\usepackage{mathrsfs}
\usepackage{makeidx}
\newtheorem{prob}{Problema}
\addtolength{\hoffset}{-2cm}
\addtolength{\textwidth}{4cm}
\addtolength{\voffset}{-2.5cm}
\addtolength{\textheight}{5cm}
\pagestyle{empty}


\begin{document}


\begin{flushright}
{\large\textbf{Tarea 3} \textnormal{Cova Pacheco Felipe de Jes\'us}, \textnormal{Domingo 6 de septiembre de 2015}}
\end{flushright}

\begin{prob} %[Fuente, capítulo, ejercicio.]
Demuestra o da un contraejemplo.\\
a. Si $p$ es un primo, entonces $p^2 + 1$ es primo.\\
b. Si $p$ es un primo, entonces $p+2$ es primo.\\
c. Si $p$ es un primo tal que $p|ab$, entonces $p|a$ o $p|b$.\\
d.  Hay primos de la forma $n! + 1$.
\end{prob}

\begin{proof}
a) Falso, si $p=3$, entonces $p^2 + 1 = 10$ que no es primo\\\

b) Falso, si $p = 13$ entonces $p+2 = 15$ no es primo\\\

c) Verdadero, si $a = P_1^{n_1} \cdots P_t^{n_i}$ y $b = P_1^{m_1} \cdots P_t^{m_t}$\\
entonces $ab = P_1^{n_1+m_1} \cdots P_t^{n_t + m_t}$ y como $P|ab$\\
entonces $P = P_i$ para alguna $i \in \{1,...,t\}$ y $n_i+m_i \geq 1$\\
lo que implica que $n_i \geq 1$ o $m_i \geq 1$ es decir $p|a$ o $p|b$.\\\

d)Verdadero, $3! + 1 = 6 + 1 = 7$ que es primo.\\\\
\end{proof}


\begin{prob} %[Fuente, capítulo, ejercicio.]
Encuentra nueve n\'umeros compuestos consecutivos.
\end{prob}

\begin{proof}
Consideramos los n\'umeros consecutivos $10!+2, 10! + 3, 10! + 4, 10! + 5, 10! + 6, 10! + 7, 10! + 8, 10! + 9, 10! + 10$.\\
Cada uno de ellos es compuesto, pues $i|10!+i$ para toda $i \in \{2, \cdots, 10\}$.\\\\
\end{proof}



\begin{prob} %[Fuente, capítulo, ejercicio.]
Cu\'antos primos hay menores o iguales a $131$?
\end{prob}

\begin{proof}
Hay $32$ primos, a saber: $2,3,5,7,11,17,19,23,29,31,37,41,43,47,53,59,61,67,71,\\73,79,83,89,87,101,103,107,109,113,127$ y $131$.\\\\
\end{proof}



\begin{prob} %[Fuente, capítulo, ejercicio.]
Determina cu\'ales de las siguientes ecuaciones diofantinas tienen soluci\'on. En el caso de tenerla, encuentra la soluci\'on general.\\
a. $14x + 16y = 15$\\
b. $15x + 21y = 39$\\
c. $1776x + 1976y = 4152$\\
d. $1076x + 2076y = 1155$
\end{prob}

\begin{proof}
a) $14x + 16y = 15$ no tiene soluci\'on, pues $(14,16) = 2$ y $2 \nmid 15$.\\\

b) $15x + 121y = 39$ si tiene soluci\'on, pues $(15, 121) = 1$.\\
encontremos una soluci\'on particular.\\
$121 = 8*15+1$, por lo que $1 = 121 - 8*15$, multiplicando por 39 en ambos lados llegamos a 
$39 = 121 * 39 - 15(39*8)$\\
Entonces $X_0 = 39$ y $Y_0 = 39*8$ son soluciones particulares,\\
as\'i $x = 39 + 121t$ y $y = 39*8 - 15t$, $t \in \mathbb{Z}$ son las soluciones generales.\\\

c) $1776x + 1976y = 4152$\\
$1976 = 1*1776 + 200$\\
$1776 = 8*200 + 176$\\
$200 = 1*176 + 24$\\
$176 = 7*24 + 8$\\
$24 = 8*3 + 0$\\
Entonces $(1776, 1976) = (24,8) = 8$ y $8|4152$\\
Por lo tanto La ecuaci\'on tiene soluci\'on\\
Encontremos una soluci\'on particular\\
$8 = 176 - 7*24 = 200-24-7*24 = 200-8*24 = 200-8(200-176)$\\
$= -7*200+8*176 = -7*200+8(1776-8*200) = -71*200+8*1776$\\
$= -71*(1976-1776)+8*1776 = -71*1976+79*1776$\\
As\'i $8= -71(1976)+79(1776)$, multiplicando por $519$ de ambos lados tenemos que\\
$4152 = -36849(1976) + 41001(1776)$\\
Entonces $X_0= -36849$ y $Y_0= 41001$ son soluciones particulares y as\'i:\\
$x= -36849 + \frac{1976}{8}(t)$ y $y= 41001 - \frac{1776}{8}(t), t \in \mathbb{Z}$\\
son las soluciones generales.\\\

d) $1076x + 2076y = 1155$\\
$2076 = 1*1076 + 1000\\
1076 = 1*1000 + 76\\
1000 = 13*76 + 12\\
76 = 6*12 + 4\\
12 = 3*4 + 0\\$
Entonces\\
$(1076, 2076) = (12,4) = 4$. y $4 \nmid 1155$\\
Por lo tanto La ecuaci\'on no tiene soluci\'on.\\\\
\end{proof}

\begin{prob} %[Fuente, capítulo, ejercicio.]
Una alcanc\'ia contiene monedas de $5, 10$ y $25$ centavos. En total contiene 4 d\'olares y hay el doble de monedas de $25$ centavos que monedas de $10$ centavos. Encuentra el n\'umero de combinaciones posibles si hay m\'as monedas de $25$ centavos que de $5$ centavos.
\end{prob}

\begin{proof}
Sean $x, y$ y $z$ la cantidad de monedas de $5, 10$ y $25$ centavos respectivamente.\\
Entonces $400 = 5x + 10y + 25z$ adem\'as\\
sabemos que $x = 2y$ por lo que\\
$400 = 5x + 10y + 50y = 5x + 60y$\\
como $(5,60) = 5$ y $5|900$, la ecuaci\'on tiene soluci\'on\\
para encontrar una soluci\'on particular expresamos $(5, 60)$ como combinaci\'on lineal de $5$ y $60$, es decir:\\
$5 = -11*5 + 60$\\
Multiplicamos por $80$, entonces\\
$400 = -880*5 + 80*60$\\
Entonces $X_0= -880$ y $Y_0= 80$ son una soluci\'on particular de la ecuaci\'on $400 = 5x + 60y$.\\
Pero s\'olo las soluciones positivas nos dar�n combinaciones que resuelvan el problema\\
Las soluciones generales de la ecuaci\'on est\'an dadas por:\\
$x=-880 + \frac{60}{5}(t) = -880 + 12t$ y $y=80 - \frac{5}{3}(t) = 80 - t$\\
Entonces para que $x \geq 0$ necesitamos que $t \geq 74$ y para que $y \geq 0$ necesitamos\\
que $t \geq 80$. Entonces las posibles combinaciones est\'an dadas por:\\
$x=-880 + 12t$ y $y=80 - t$, $t \in \{74, 75, \cdots. 80\}$\\\\
\end{proof}

\begin{prob} %[Fuente, capítulo, ejercicio.]
Encuentra el residuo de $4^{117}$ cuando es dividido por $15$.
\end{prob}

\begin{proof}
Tenemos que $4 \equiv 11$ (mod 15), entonces\\
$4^{117} \equiv 11^{117}$ (mod 15)\\
y como $11^2 = 121 \equiv 1$ (mod 15) podemos escribir\\
$11^{117} = 11^{2*58+1} = (11^2)^{58+1} = (11^2)^{58}*11$\\
As\'i $4^{117} \equiv 11^{117} \equiv (11^2)^{58}*11 \equiv (1)^{58}*11 \equiv 11$ (mod 15)\\\\
\end{proof}


\begin{prob} %[Fuente, capítulo, ejercicio.]
Si hoy es viernes, qu\'e d\'ia ser� en 1976 d\'ias?
\end{prob}

\begin{proof}
Supongamos que hoy es el d\'ia $x$, como viernes\\
es el 5to d\'ia de la semana, tenemos que\\
$x \equiv 5$ (mod 7)\\
por otro lado $1976 \equiv 2$ (mod 7) entonces $x + 1976 \equiv 0$ (mod 7)\\
Por lo que el d\'a $x+1976$ es el s\'eptimo de la semana,\\
es decir si hoy es viernes, en $1976$ d\'ias es domingo.\\\\
\end{proof}


\begin{prob} %[Fuente, capítulo, ejercicio.]
Encuentra el residuo de $1! + 2! + 3! + ... + 1000!$ al ser dividido por:\\
a. $12$\\
b. $11$
\end{prob}

\begin{proof}
a) Notamos que cuando $k \geq 4$, se tiene que $12|k!$\\
y as\'i $k! \equiv 0$ (mod 12)\\
entonces $1! + 2! + \cdots + 1000! \equiv 1! + 2! + 3!$ (mod 12) $\equiv 1 + 2 +6$ (mod 12) $\equiv 9$ (mod 12)\\
Por lo tanto cuando $1! + 2! + \cdots + 1000!$ es dividido por $12$, el residuo es $9$.\\\

b) Notamos que cuando $k \geq 11$ tenemos que $k! \geq 0$ (mod 11)\\
as\'i $1! + \cdots + 1000! \equiv 1! + 2! + \cdots + \10!$ (mod 11)\\
$\equiv 1 + 2 + 6 + 24 + 120 + 740 + 5040 + 40320 + 362880 + 3628800$ (mod 11)\\
$\equiv 1 + 2 + 6 + 2 + 10 + 3 + 2 + 5 + 1 + 10$ (mod 11)\\
$\equiv 9$ (mod 11).\\\\
\end{proof}


\begin{prob} %[Fuente, capítulo, ejercicio.]
Encuentra el m\'inimo residuo de $(p-1)!$ m\'odulo $p$ para $p = 3,5,7$.
\end{prob}

\begin{proof}
Para $P=3$\\
$(P-1)! = 2! = 2$ y $3 \equiv 2$ (mod 3)\\\

Para $P=5$\\
$(P-1)! = 4! = 24$ y $24 \equiv 4$ (mod 5)\\\

Para $P=7$\\
$(P-1)! = 6! = 720$ y $720 \equiv 6$ (mod 7)
\end{proof}


\begin{prob} %[Fuente, capítulo, ejercicio.]
Demuestra que si $2a \equiv 0$ (mod p) y $p$ es un primo impar, entonces $a \equiv 0$ (mod p).
\end{prob}

\begin{proof}
Suponemos que $2a \equiv 0$ (mod P) y $P$ es impar.\\
Por lo anterior $P|2a$ y como $P$ es impar\\
tenemos que $p \geq 3$, por lo que $P$ no puede dividir a $2$\\
entonces $P|a$ y as\'i $a \equiv 0$ (mod p)
\end{proof}

\end{document}




